% Options for packages loaded elsewhere
% Options for packages loaded elsewhere
\PassOptionsToPackage{unicode}{hyperref}
\PassOptionsToPackage{hyphens}{url}
\PassOptionsToPackage{dvipsnames,svgnames,x11names}{xcolor}
%
\documentclass[
  spanish,
  12pt,
  letterpaper,
]{article}
\usepackage{xcolor}
\usepackage[margin=1in]{geometry}
\usepackage{amsmath,amssymb}
\setcounter{secnumdepth}{5}
\usepackage{iftex}
\ifPDFTeX
  \usepackage[T1]{fontenc}
  \usepackage[utf8]{inputenc}
  \usepackage{textcomp} % provide euro and other symbols
\else % if luatex or xetex
  \usepackage{unicode-math} % this also loads fontspec
  \defaultfontfeatures{Scale=MatchLowercase}
  \defaultfontfeatures[\rmfamily]{Ligatures=TeX,Scale=1}
\fi
\usepackage{lmodern}
\ifPDFTeX\else
  % xetex/luatex font selection
  \setmainfont[]{Times New Roman}
\fi
% Use upquote if available, for straight quotes in verbatim environments
\IfFileExists{upquote.sty}{\usepackage{upquote}}{}
\IfFileExists{microtype.sty}{% use microtype if available
  \usepackage[]{microtype}
  \UseMicrotypeSet[protrusion]{basicmath} % disable protrusion for tt fonts
}{}
\usepackage{setspace}
\makeatletter
\@ifundefined{KOMAClassName}{% if non-KOMA class
  \IfFileExists{parskip.sty}{%
    \usepackage{parskip}
  }{% else
    \setlength{\parindent}{0pt}
    \setlength{\parskip}{6pt plus 2pt minus 1pt}}
}{% if KOMA class
  \KOMAoptions{parskip=half}}
\makeatother
% Make \paragraph and \subparagraph free-standing
\makeatletter
\ifx\paragraph\undefined\else
  \let\oldparagraph\paragraph
  \renewcommand{\paragraph}{
    \@ifstar
      \xxxParagraphStar
      \xxxParagraphNoStar
  }
  \newcommand{\xxxParagraphStar}[1]{\oldparagraph*{#1}\mbox{}}
  \newcommand{\xxxParagraphNoStar}[1]{\oldparagraph{#1}\mbox{}}
\fi
\ifx\subparagraph\undefined\else
  \let\oldsubparagraph\subparagraph
  \renewcommand{\subparagraph}{
    \@ifstar
      \xxxSubParagraphStar
      \xxxSubParagraphNoStar
  }
  \newcommand{\xxxSubParagraphStar}[1]{\oldsubparagraph*{#1}\mbox{}}
  \newcommand{\xxxSubParagraphNoStar}[1]{\oldsubparagraph{#1}\mbox{}}
\fi
\makeatother


\usepackage{longtable,booktabs,array}
\usepackage{calc} % for calculating minipage widths
% Correct order of tables after \paragraph or \subparagraph
\usepackage{etoolbox}
\makeatletter
\patchcmd\longtable{\par}{\if@noskipsec\mbox{}\fi\par}{}{}
\makeatother
% Allow footnotes in longtable head/foot
\IfFileExists{footnotehyper.sty}{\usepackage{footnotehyper}}{\usepackage{footnote}}
\makesavenoteenv{longtable}
\usepackage{graphicx}
\makeatletter
\newsavebox\pandoc@box
\newcommand*\pandocbounded[1]{% scales image to fit in text height/width
  \sbox\pandoc@box{#1}%
  \Gscale@div\@tempa{\textheight}{\dimexpr\ht\pandoc@box+\dp\pandoc@box\relax}%
  \Gscale@div\@tempb{\linewidth}{\wd\pandoc@box}%
  \ifdim\@tempb\p@<\@tempa\p@\let\@tempa\@tempb\fi% select the smaller of both
  \ifdim\@tempa\p@<\p@\scalebox{\@tempa}{\usebox\pandoc@box}%
  \else\usebox{\pandoc@box}%
  \fi%
}
% Set default figure placement to htbp
\def\fps@figure{htbp}
\makeatother



\ifLuaTeX
\usepackage[bidi=basic]{babel}
\else
\usepackage[bidi=default]{babel}
\fi
\ifPDFTeX
\else
\babelfont{rm}[]{Times New Roman}
\fi
% get rid of language-specific shorthands (see #6817):
\let\LanguageShortHands\languageshorthands
\def\languageshorthands#1{}


\setlength{\emergencystretch}{3em} % prevent overfull lines

\providecommand{\tightlist}{%
  \setlength{\itemsep}{0pt}\setlength{\parskip}{0pt}}



 
\usepackage[style=apa,]{biblatex}
\addbibresource{references.bib}


\usepackage{booktabs}
\usepackage{longtable}
\usepackage{array}
\usepackage{multirow}
\usepackage{wrapfig}
\usepackage{float}
\usepackage{colortbl}
\usepackage{pdflscape}
\usepackage{tabu}
\usepackage{threeparttable}
\usepackage{threeparttablex}
\usepackage[normalem]{ulem}
\usepackage{makecell}
\usepackage{xcolor}
\usepackage{setspace}
\usepackage{float}
\usepackage{indentfirst}
\makeatletter
\@ifpackageloaded{caption}{}{\usepackage{caption}}
\AtBeginDocument{%
\ifdefined\contentsname
  \renewcommand*\contentsname{Tabla de contenidos}
\else
  \newcommand\contentsname{Tabla de contenidos}
\fi
\ifdefined\listfigurename
  \renewcommand*\listfigurename{Listado de Figuras}
\else
  \newcommand\listfigurename{Listado de Figuras}
\fi
\ifdefined\listtablename
  \renewcommand*\listtablename{Listado de Tablas}
\else
  \newcommand\listtablename{Listado de Tablas}
\fi
\ifdefined\figurename
  \renewcommand*\figurename{Figura}
\else
  \newcommand\figurename{Figura}
\fi
\ifdefined\tablename
  \renewcommand*\tablename{Tabla}
\else
  \newcommand\tablename{Tabla}
\fi
}
\@ifpackageloaded{float}{}{\usepackage{float}}
\floatstyle{ruled}
\@ifundefined{c@chapter}{\newfloat{codelisting}{h}{lop}}{\newfloat{codelisting}{h}{lop}[chapter]}
\floatname{codelisting}{Listado}
\newcommand*\listoflistings{\listof{codelisting}{Listado de Listados}}
\makeatother
\makeatletter
\makeatother
\makeatletter
\@ifpackageloaded{caption}{}{\usepackage{caption}}
\@ifpackageloaded{subcaption}{}{\usepackage{subcaption}}
\makeatother
\usepackage{bookmark}
\IfFileExists{xurl.sty}{\usepackage{xurl}}{} % add URL line breaks if available
\urlstyle{same}
\hypersetup{
  pdftitle={Recognizing inequalities in the Chilean academia: A longitudinal analysis of how Gender, Social Class, and Geographic Origins affects Scientific Careers},
  pdfauthor={Roxana Chiappa; Paula Clasing; Carlos Anríquez},
  pdflang={es},
  pdfkeywords={geographical disparities, interseccionality, PHD
education, academic inequalities},
  colorlinks=true,
  linkcolor={blue},
  filecolor={Maroon},
  citecolor={Blue},
  urlcolor={Blue},
  pdfcreator={LaTeX via pandoc}}


\title{Recognizing inequalities in the Chilean academia: A longitudinal
analysis of how Gender, Social Class, and Geographic Origins affects
Scientific Careers}
\author{Roxana Chiappa \and Paula Clasing \and Carlos Anríquez}
\date{2025-12-24}
\begin{document}
\maketitle

\renewcommand*\contentsname{Tabla de contenidos}
{
\hypersetup{linkcolor=}
\setcounter{tocdepth}{3}
\tableofcontents
}

\setstretch{1.5}
\section{Introduction}\label{introduction}

One of the central promises of scientific institutions lies in their
supposed meritocratic character: a reward system where scientific
outcomes are evaluated based mainly on intellectual merit. Yet this
ideal has been widely contested. Throughout the 20th and 21st centuries,
research has shown that the social stratification present in broader
society permeates the organization and prestige hierarchies
(references). As a result, academic and scientific institutions not only
reflect existing inequalities but often actively reproduce them.

Among these, gender has been the most extensively studied societal
structure for understanding how stratification and inequality are
(re)produced through both institutionalized and informal mechanisms
within academic and scientific institutions (xxxx). Hundreds of
scholarly works have examined how gender structures---often
conceptualized in binary terms---constrain women's access to and
advancement within academia across different contexts and roles (xxxx).
In many countries, women's representation has increased at the
undergraduate level, including in math-intensive fields historically
dominated by men. However, their presence declines sharply at senior
ranks, echoing what has been widely described as the ``leaky pipeline.''

While gender remains central to understanding inequality in scientific
careers, we argue for widening the analytical lens to include other
persistent structures---particularly social class and geographic
origin---that have shaped the educational trajectories of those who are,
or will become, the next generation of scientists. These factors have
been widely analyzed in the broad social stratification literature
(xxxxxx), particularly for their impact on access to prestigious
educational institutions. Yet they remain largely overlooked in studies
of scientific careers; perhaps due to the difficulty of accessing these
data and/or the assumption that class and geographic origin become
imperceptible at the light of their scientific productivity (xxxx).

This article addresses that gap by analyzing three cohorts of Chilean
researchers (N = 2,436) who received a competitive government-funded
doctoral fellowship between 2014 and 2016. Chile was selected as the
national case for three key reasons. First, the country significantly
expanded its doctoral fellowship programs after 2008, offering funding
for studies both within Chile and abroad. Second, Chile exhibits high
levels of social class and geographic inequality---patterns that
characterize the higher education systems of many countries across Latin
America. Third, it offers access to a unique longitudinal dataset that
includes researchers' high school and regional background, enabling us
to examine the effects of gender, class, and geographic origin across
three key academic milestones: (1) the selection and unequal chances of
entering a prestigious undergraduate institution, (2) the pursuit of
doctoral studies, and (3) scientific productivity in the six years
following the fellowship award. Using this dataset, we applied a path
model analysis to estimate the direct and indirect effects of gender,
social class, and geographic origin across these three moments.

While gendered patterns of scientific productivity align with existing
literature, our findings complicate linear models of stratification.
Researchers from working-class backgrounds produce research outputs at
rates statistically similar to those of their upper-class peers. Even
more unexpectedly, researchers who grew up in less economically and
culturally developed regions outperform those from the wealthier,
metropolitan center. These findings suggest that commonly held
assumptions about advantage and academic performance do not fully
account for the lived realities---and possible exceptional
adaptations---of those who enter academia from structurally marginalized
positions.

\section{Conceptual framework: stratification on
sciences.}\label{conceptual-framework-stratification-on-sciences.}

\subsection{Geographical disparities in academic
trayectories}\label{geographical-disparities-in-academic-trayectories}

\subsection{Socioeconomic status}\label{socioeconomic-status}

\subsection{Disciplinary differences in scientific
production}\label{disciplinary-differences-in-scientific-production}

\subsection{Gender and scientific
production}\label{gender-and-scientific-production}

\subsection{Scolarship selection
process}\label{scolarship-selection-process}

\section{Method}\label{method}

\subsection{Data and sample}\label{data-and-sample}

The database is a longitudinal reconstruction of the academic
trajectories of three cohorts of Chilean researchers (N = 2436) who
received a doctoral scholarship between 2014 and 2016. It includes
information on the socioeconomic level of the high school they attended,
the region of the institution, the accreditation level of their
undergraduate program, the prestige of the doctoral university, the
postgraduate area, and the number of publications produced up to six
years after the scholarship was awarded.

\subsection{Variables}\label{variables}

Table 1 shows the variables used in the analysis. These summarize the
researchers' trajectory from secondary education to scientific
productivity after the doctorate. The analysis employed three
sociodemographic variables: sex, socioeconomic status, and region. To
measure socioeconomic status, the socioeconomic classification of each
researcher's high school served as a proxy, while geographic origin was
determined by the educational institution's region.

Undergraduate prestige was measured through the accreditation level of
the researcher's university as determined by the National Education
Commission. For doctoral prestige, we calculated the average between the
scores in the Shanghai and QS international rankings, expressed as a
Z-score. Finally, scientific productivity was measured through the
number of publications, standardized based on the average publications
in the researcher's area.

\begingroup\fontsize{11}{13}\selectfont

\begin{longtable}[t]{lr}
\caption{\label{tab:tabla-variables}Table 1. Descriptive Statistics}\\
\toprule
Category & Statistic\\
\midrule
\addlinespace[0.3em]
\multicolumn{2}{l}{\cellcolor[HTML]{f0f0f0}{\textbf{Gender}}}\\
\addlinespace[0.3em]
\multicolumn{2}{l}{\cellcolor[HTML]{f0f0f0}{\textbf{National PhD}}}\\
\addlinespace[0.3em]
\multicolumn{2}{l}{\cellcolor[HTML]{f0f0f0}{\textbf{PhD Prestige (Z)}}}\\
\addlinespace[0.3em]
\multicolumn{2}{l}{\cellcolor[HTML]{f0f0f0}{\textbf{Publication before PhD}}}\\
\addlinespace[0.3em]
\multicolumn{2}{l}{\cellcolor[HTML]{f0f0f0}{\textbf{Publications after PhD (Z)}}}\\
\addlinespace[0.3em]
\multicolumn{2}{l}{\cellcolor[HTML]{f0f0f0}{\textbf{Region}}}\\
\addlinespace[0.3em]
\multicolumn{2}{l}{\cellcolor[HTML]{f0f0f0}{\textbf{Socio-economic Status}}}\\
\addlinespace[0.3em]
\multicolumn{2}{l}{\cellcolor[HTML]{f0f0f0}{\textbf{Undergraduate Prestige}}}\\
\hspace{1em}\hspace{1em}\hspace{1em}\hspace{1em}\hspace{1em}\hspace{1em}\hspace{1em}\hspace{1em}\cellcolor{gray!10}{Femenino} & \cellcolor{gray!10}{1010 (41.5\%)}\\
\hspace{1em}\hspace{1em}\hspace{1em}\hspace{1em}\hspace{1em}\hspace{1em}\hspace{1em}\hspace{1em}Masculino & 1426 (58.5\%)\\
\hspace{1em}\hspace{1em}\hspace{1em}\hspace{1em}\hspace{1em}\cellcolor{gray!10}{Alto} & \cellcolor{gray!10}{849 (34.9\%)}\\
\hspace{1em}\hspace{1em}\hspace{1em}\hspace{1em}Bajo & 473 (19.4\%)\\
\hspace{1em}\cellcolor{gray!10}{Medio} & \cellcolor{gray!10}{663 (27.2\%)}\\
Medio alto & 451 (18.5\%)\\
\cellcolor{gray!10}{Otra región} & \cellcolor{gray!10}{720 (29.6\%)}\\
Región Metropolitana & 1190 (48.9\%)\\
\cellcolor{gray!10}{Valparaiso o Concepción} & \cellcolor{gray!10}{525 (21.6\%)}\\
Acreditacion 4 & 188 (7.7\%)\\
\cellcolor{gray!10}{Acreditacion 5} & \cellcolor{gray!10}{373 (15.3\%)}\\
Baja acreditacion o no acreditada & 106 (4.4\%)\\
\cellcolor{gray!10}{Excelencia} & \cellcolor{gray!10}{812 (33.3\%)}\\
Investigacion & 957 (39.3\%)\\
\cellcolor{gray!10}{Sin publicaciones} & \cellcolor{gray!10}{1742 (71.5\%)}\\
Tiene publicaciones & 694 (28.5\%)\\
\cellcolor{gray!10}{Max.} & \cellcolor{gray!10}{4.2}\\
Mean & 0.04\\
\cellcolor{gray!10}{Min.} & \cellcolor{gray!10}{-1.35}\\
SD & 0.96\\
\cellcolor{gray!10}{Chile} & \cellcolor{gray!10}{1523 (62.5\%)}\\
Extranjero & 913 (37.5\%)\\
\cellcolor{gray!10}{Max.} & \cellcolor{gray!10}{28.46}\\
Mean & 0\\
\cellcolor{gray!10}{Min.} & \cellcolor{gray!10}{-0.84}\\
SD & 1\\
\bottomrule
\end{longtable}
\endgroup{}

\subsection{Analytic Strategy}\label{analytic-strategy}

\section{Findings}\label{findings}

Z scopus después -\textgreater{} terdoc (0.323) -\textgreater{} fem
(-0.191) -\textgreater{} tipuban (-0.309) -\textgreater{} medio alto
(-0.127) -\textgreater{} medio (-0.106) -\textgreater{} metval (-0.141)
-\textgreater{} doccl (-0.131)

Prestigio -\textgreater{} medio alto (-0.127) -\textgreater{} medio
(-0.205) -\textgreater{} bajo (-0.331) -\textgreater{} región (0.134)
-\textgreater{} acred (0.09) -\textgreater{} doccl(-1.073)

Tipuban -\textgreater{} Sexo (-0.038) -\textgreater{} metval (-0.041)

Acred -\textgreater{} medio alto (-0.388) -\textgreater{} medio (-0.366)
-\textgreater{} bajo (-0.741)

\section{Conclusions}\label{conclusions}

\subsection{Referencias}\label{referencias}

\printbibliography[heading=none]





\end{document}
